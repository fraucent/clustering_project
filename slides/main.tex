\documentclass{beamer}
\usetheme{metropolis}
\usepackage[utf8]{inputenc}
\usepackage{amsmath}
\usepackage{amsfonts}
\usepackage{amssymb}
\usepackage{graphicx}
\usepackage[french]{babel}
\usepackage{iflang}
%\usepackage[dvipsnames]{xcolor}
\usepackage{listings}
\usepackage{lipsum}
\usepackage{tikz}
\usepackage{boolexpr}
\usepackage{kpfonts}
\usepackage{caption}
\usepackage{wrapfig}
\usepackage{tikz}
\usepackage{chngcntr}
\usepackage{verbatim}
\usepackage[labelformat=empty]{caption}

\title{Implémentation et test de GRGPF sur des BoW}
\date{14 décembre 2015}
\author{Guillaume Derval, François Raucent}
\begin{document}
  \maketitle
  \metroset{block=fill}
  \tableofcontents
  \section{Implémentation de GRGPF}
  \begin{frame}
  	\frametitle{Rappel de GRGPF}
  	\begin{itemize}
  		\item Hierarchical clustering + point assignement
  		\item Maintien un arbre contenant tout les clusters actuels dans ses feuilles
  		\item Les noeuds internes contiennent des échantillons des clustroid contenu dans leur sous-arbre
  		\item Représentation spécifique d'un cluster
  		\item Assigne les points dans ces clusters un à un
  		\item Si cluster trop large $\rightarrow$ split
  		\item Si plus de mémoire $\rightarrow$ merge
  	\end{itemize}
  \end{frame}
  \begin{frame}
  	\frametitle{Choix initiaux}
  	\begin{itemize}
  		\item Python
  		\item Indépendant de la représentation des données/de la distance choisie
  		\item Proche de la description originale de GRGPF/BUBBLE
  		\item \alert{Non-balancé}
  		\item \alert{N'optimise pas l'emplacement mémoire}
  	\end{itemize}
  \end{frame}
  
 	%\begin{frame}
 	%	\frametitle{Choix d'implémentation}
 	%	\begin{itemize}
 	%		\item Ni le livre ni l'article d'origine ne décrivent complétement GRGPF
 	%		\item Certains points sont en partie laissés à l'appréciation de la personne qui implémente l'algorithme:
	% 			\begin{itemize}
	% 				\item Comment split en deux un cluster?
	% 				\item Quand il n'y a plus de mémoire, quels et combien de cluster faut-il fusionner?
	% 				\item Comment choisir les échantillons à mettre dans les noeuds internes de l'arbre?
	% 				\item Quand/comment recalculer une représentation de cluster?
	% 			\end{itemize}
 	%	\end{itemize}
 	%\end{frame}
 	
 	\begin{frame}
 		\frametitle{Représentation d'un cluster}
 		\begin{itemize}
 			\item Clustroid
 			\item \texttt{ROWSUM} du Clustroid
 			\item $k$ points les plus proches du clustroid
 			\item Leurs \texttt{ROWSUM}
 			\item Idem pour les $k$ points les plus éloignés.
 			\item Particularité: les $2k$ points sont \alert{triés} pour avoir $O(\log(k))$ sur la majorité des opérations
 		\end{itemize}
 		
 		Quatre opérations:
 		\begin{enumerate}
 			\item \texttt{get\_radius()}
 			\item \texttt{add\_point(p)}
 			\item \texttt{recompute()}
 			\item \texttt{split()}
		\end{enumerate}
 	\end{frame}
 	
 	\begin{frame}
 		\frametitle{Noeuds feuilles}
 		Contiennent simplement les clusters.
 		
 		Trois opérations:
 		\begin{enumerate}
 			\item \texttt{add\_point(p)}: trouve le cluster qui a le clustroid le plus proche du point $p$ et appelle \texttt{add\_point(p)} sur celui-ci
 			\item \texttt{add\_cluster(c)}: trouve le cluster le plus proche du clustroid de $c$, et merge ces clusters si besoin. Sinon, ajoute simplement le nouveau cluster.
 			\item \texttt{split()}: splitte ce noeud en deux, en répartissant les clustroid
 			\item \texttt{recompute()}: recalcule toutes les représentations de clusters contenues dans ce noeud
 		\end{enumerate}
 	\end{frame}
 	
 	\begin{frame}
 		\frametitle{Noeuds internes}
 		Contiennent un échantillon des clustroids du sous-arbre dont il est la racine.
 		
 		Trois opérations:
 		\begin{enumerate}
 			\item \texttt{add\_point(p)}: trouve le noeud enfant qui a le clustroid le plus proche (parmi l'échantillon) du point $p$ et appelle \texttt{add\_point(p)} sur celui-ci
 			\item \texttt{add\_cluster(c)}: idem, pour un cluster
 			\item \texttt{split()}: splitte ce noeud en deux, en répartissant les clustroid
 			\item \texttt{recompute()}: recalcule toutes les représentations de clusters contenues dans ce noeud
 		\end{enumerate}
 	\end{frame}
 	
 	\begin{frame}
 		\frametitle{Echantionnage des clustroids}
 		Soit un noeud avec $n$ enfants, chaque enfant $i$ contenant $m_i$ clustroids (échantillonés ou pas), et une constante $S$, la taille de l'échantillon que l'on souhaite.
 		
 		On prendra $\max(\lfloor\frac{m_i*S}{\sum_{j=0}^n m_j}\rfloor,1)$ clustroid de chaque enfant $i$, au hasard (uniformement).
 	\end{frame}
 	
 	\begin{frame}
 		\frametitle{Mise à jour des échantillons}
 		\begin{itemize}
 			\item Lors du split d'un noeud enfant, mise à jour des échantillon du parent
 			\item Recalcul périodique des échantillons de tout les noeuds
 		\end{itemize}
 	\end{frame}
 	
 	\begin{frame}
 		\frametitle{Implémentation de split - Cluster}
 		Peu défini par l'article original. On utilise ici une heuristique proposée dans l'algorithme \texttt{BIRCH}.
 		\begin{enumerate}
 			\item Remettre tout les points du cluster en mémoire
 			\item Prendre la paire de point la plus éloignée
 			\item En faire deux clustroid temporaires
 			\item Pour chaque point restant, l'assigner au clustroid le plus proche
 			\item Recalculer les représentations des clusters résultants
 		\end{enumerate}
 	\end{frame}
 	
 	\begin{frame}
 		\frametitle{Implémentation de split - Noeud}
 		De nouveau, non défini dans l'article original. Idée toujours prise de \texttt{BIRCH}.
 		\begin{enumerate}
 			\item Prendre le sample de clustroid
 			\item Trouver les deux clustroids les plus éloignés
 			\item Répartir les clustroid en deux groupes en fonctions de leur distance
 			\item Créer deux noeuds à partir de cela
 		\end{enumerate}
 	\end{frame}
 	
 	\begin{frame}
 		\frametitle{Merge}
 		\begin{itemize}
 			\item But: réduire la mémoire
 			\item Augmentation du threshold de split
 			\item Recréation entière de l'arbre, en traitant les clusters comme les points
 			\item Calcul des représentation mergée présentée dans le cours
 		\end{itemize}
 	\end{frame}
 	
 	\section{Tests et résultats}
 	\section{Conclusion}
\end{document}